\section{Evaluation}
\label{sec:evaluation}

In our evaluation, we have two aims:\\
(1) to demonstrate the performance of our implementation against the 
state-of-the-art XML database BaseX on a single computer and \\ 
(2) to explore the scalability of our implementation on processing 
very large XML documents in a parallel manner.

\subsection{Settings}

\subsubsection{Datasets and XPath Queries} 

We used XMark\footnote{\url{http://www.xml-benchmark.org/}} xmlgen to
generate XML document for the experiment. The XMark xmlgen takes an
float number \emph{f} to determine the size of output dataset. In our
experiment, we set f to 160 and an 18.54 GiB XML document xmark160 was
generated, which has 267 M element nodes, 61.3 M attribute nodes and
188 M content nodes, totally 516.3 M nodes. We used 7 queries Q1 -- Q7
to evaluate our implementation. 

\subsubsection{Experimental set-up} 

The EC2 instance type We used in our experiment was m5.2xlarge, which
is based on E5-2670 v2 (Ivy Bridge), equipped with 32 GB of memory.
The instance type runs Amazon Linux AMI 2018.03.0 (HVM), which is an
EBS-backed, AWS-supported image that supports Java by default.  The
Java version we used was "1.7.0\_181",  OpenJDK Runtime Environment
(amzn-2.6.14.8.80.amzn1-x86\_64 u181-b00) OpenJDK 64-Bit Server VM
(build 24.181-b00, mixed mode).

\subsubsection{BaseX}

The version of BaseX we used was 8.6.7, which was the last version of
BaseX that requires java 1.7. We ran it in server/client mode. The
database was set in main memory mode and turned text indexing off. 


\subsection{Single Instance Query}

The first experiment was conducted to compare our BFS with BaseX. We used the
following expression to return a list of PRE-values, as is often used in common
applications.    
\verb|for $node in db:open('xmark160')$query return db:node-pre($node)|, 
where \$query represents an XPath query.

For all the severn queries, our BFS outperformances BaseX.