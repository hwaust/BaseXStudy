\documentclass{paper}

\begin{document}
	
	\section{A Scalable XML Indexing Method Using MapReduce}	% NO.1
	
	\subsection{Background}
	
	Shifting theories, methods etc from stand-along computer to cloud 
	parallel computing is very difficult. 
	
	\subsection{Technical Problem}
	
	How to make the authors' previous study CIS-X (A Compressed Index
	Scheme for Efficient Query Evaluation of XML Documents) designed for a
	single computer work in parallel using multiple computers.
	
	\subsection{Technical claim}
	
	They redesigned an XML indexing method, called CIS-X to use MapReduce
	implemented in Hadoop, that can efficiently handle large XML documents
	through cloud parallel computing.  
	
	\subsection{Discussion}

	For clarity, the target parallel version of CIS-X is denoted 
	as CIS-XP to distinguish with the original CIS-X in the following discussion.
	The proposed cloud-based CIS-XP is a scalable design and is able to
	build indexes and process queries on large XML files through parallel
	processing. 
	
	
	The approach has two phases: index creating phase and query evaluating
	phase (which is the same as our index-based representation of partial
	tree).  In the first phase, one difficulty is that the CIS-X keeps
	index in memory, which encounters the out-of-memory problem while
	dealing with a large XML file because of the limits of memory and
	address space. Another difficulty is encountered when an XML file is 
	large. Thus, it is necessary to be split into several parts and 
	input into MapReduce one at a time. They used SAX to parse an input XML 
	document for minimizing memory required. 
	
	To identify nodes in an XML document, 
	they designed Nid for nodes in pre-order traversal. A node is
	represented by a 4-tuple  (\emph{Type}, \emph{Tag}, \emph{Dewey}, 
	\emph{Nid-[content]}), where \emph{Type} is the type of then node, 
	including \emph{node} for element, 	\emph{value} for
	content. \emph{Tag} is the tag name of the node. \emph{Dewey} is a Dewey 
	label which is used to identify the position of the  node in the summarized
	index. Each node of XML is labeled by a Nid, as mentioned above.  If
	the node is a text node, the \emph{[content]} records the value. 
	In the query evaluation phase, the indexes can be used to accelerate 
	child and descendant axes.
	
	
	\subsection{Experiments} 
	
	The experiments on a Linux cluster with AMD Opteron Processor 6128 CPU
	2 GHz, 3 physical machines with 32G RAM, each of which has 2 virtual
	machines with 4G RAM. There are 6 virtual machines, one for Name Node,
	one for Secondary Name Node, and four for Data Nodes (worker nodes).
	The test dataset is 1.2G DBLP with 29,242,336 elements, 7,325,278
	attributes, and 26,619,477 texts.
	
	The experimental results show that there is a linear relationship
	between the file size and the execution time for index construction.
	The query processing is also very efficient and is mainly determined
	by the number of target nodes and the output file size for a query.
	
	\subsection{Conclusion}
	
	This paper is closely relevant to our study in the aspects of 
	adressed problem, XML data spliting, index-based representation.
	The different part is the implementation, which is based on SAX 
	for parsing and MapReduce for query processing.
	
	
	\section{Efficient Implementation of XPath Processor on Multi-Core CPUs} % NO.2
	
	\subsection{Background}
	
	Multiple processor cores are integrated on a single chip, making
	parallel architectures available to common users. This trend requires
	programmers to focus on problem parallelization instead of classic
	linear optimization. However, common tools for querying XML documents 
	do not exploit computational power of multi-core CPUs. 
	
	
	\subsection{Technical Problem}
	
	Straightforward implementation of XML processing has its time 
	complexity exponentially dependent on the level of predicate nesting. 
	
	\subsection{Technical Claims}
	
	We propose algorithms and XML indexing techniques which are more
	efficient and can utilize standard parallel templates. The implementation 
	is highly scalable and outperforms common XML libraries.
	
	\subsection{Discussions} 
	
	The study focuses solely on processing single query on a single
	document, which to be loaded into main memory in DOM and indexed. 
	The common parts of this study to ours are listed as below.
	
	\begin{itemize}  
		\item Each step takes node-set produced by previous step (called initial
		node-set) and generates another node-set which is used by successive
		step.
		\item When a node-set is filtered by a predicate, the predicate is executed
		recursively for every node in the set using the node, its position
		and size of the set as context values. 
		\item The index scheme is based on tagging nodes of DOM structure with three 
		integers: left (ordinal of start tag), right (ordinal of end tag), and 
		depth value (depth of the node).
	\end{itemize}
	
	To accelerate query processing, the study proposes three methods:
	
	\begin{itemize}
		\item Minimal Context Optimizations to add three tags: `flag n for context 
		node', `flag p for position' and `flag s for size' to accelerate the 
		evaluation of context nodes.
		\item Vectorization of Axes to accelerate axes. 
		\item Predicate Caches to optimize predicate evaluation.
	\end{itemize}
	
	By the above three methods, the study then uses existed paralleling 
	templates (such as parallel-for, parallel-reduce, parallel-scan, etc.) 
	and data structures (such as concurrent vector, concurrent hash-map, etc.) 
	to realize parallelization. The query processing has two phases:
	parallel query evaluating phase and sequential sub-results merging phase. 
	After processing the parallel phase, the sub-results are merged to 
	form the final results.
	
	
	\subsection{Experiments}
	
	The experiment results suggest that our implementation is highly
	scalable. The best speedup was observed at query B6 which runs 18.8×
	faster on 24 cores than on single core. 
	
	\subsection{Conclusion}
	
	This settings have some common parts as ours, such as background, 
	goal, settings etc. However, the methodology is	a bit different from ours. 
	This paper studies on how to design optimizations
	on query expressions, while ours is on how to split and represent 
	the split XML data for faster parallel processing. This paper is related 
	to our study but not very relevant.
	
	
	\section{Indexing Structured Documents with Suffix Arrays}
	
	\subsection{Background}
	
	Path indexes based on suffix trees have shown to be highly efficient
	structures when dealing with digital collection that consists of
	structured documents, since they provide a fast response to queries
	including structural requirements. The use of a suffix array as the
	underlying data storage permits a considerable reduction in space
	requirements, partially because suffix arrays are a remarkably light
	data structure and partially because they do not store redundant
	information regarding the textual content. 
	
	
	\subsection{Technical Problems}
	
	Suffix trees may be too memory demanding when processing highly
	heterogeneous documents.
	
	\subsection{Technical Claims}
	
	The authors describe how a suffix array can be used as the data
	structure which stores the structural index in a retrieval system and
	provides a virtual index of all sub-paths in the digital collection 
	to accelerate query processing with consideration of the reduction 
	of memory.
	
	\subsection{Discussion}
	
	Path and sub-path are precisely defined in the paper as:  A path in an
	XML document is the sequence of tags obtained when the document tree
	is traversed from its root to a leaf node, following a single branch.
	A subpath is the sequence of tags obtained with a partial traversal of
	the tree branch —that is, between arbitrary initial and final nodes. A
	suffix array is a remarkably compact data structure. 
	
	There are two techniques proposed in this study:
	\begin{itemize}
		\item A structural index based on a suffix array
		\item A balanced ternary search tree
	\end{itemize}
	
	
	\subsection{Experiments}
	
	All tests were evaluated using a personal computer (running at 2.8GHz
	with a 3.5GB main memory).	
	The performance of the approach has been tested with a large
	collection of XML files (approximately 3,000 files occupying a total
	of 535 MB) with TEI markup. The collection contained over 50,000
	different paths and expanded about 70 million textual nodes.
	Approximately 60,000 queries were created by a random generator of
	paths using the TEI DTD schema as input. 
	
	\subsection{Conclusion}
	
	The techniques (path index and suffix tree) used in this study are 
	different from ours. The goal and method are related to ours. Plus, 
	I consider whether we can study how path index can work with partial 
	tree since it has been proven to be effective to accelerate queries 
	and it is also possible to integrate it into our study.
	
	
	\section{Node Indexing in XML Query Optimization: A Review}
	
	\subsection{Background}
	
	Structural indexes are classified into three main  groups: Path indexing, Node
	indexing and Sequence based indexing. The focus of this paper is
	on node indexing and is a review of major four types of node indexing:
	subtree labeling, prefix-based labeling, multiplicative labeling and hybrid
	labeling. The aims of this paper is to review on some of the latest techniques
	for each node indexing group.
	
	
	\subsection{Technical claims}
	
	Each indexing techniques has its advantages and disadvantages. So choosing a
	correct indexing is critical. This paper discusses and points out how to
	correctly choose an indexing technique. Most important, this review explores and
	identifies the trends which can be useful for new researcher.
	
	
	\subsection{Discussions}
	
	This paper is a review of five indexing techniques. It demonstrates the
	techniques with concrete examples for each of them. It summarize them in Table
	2: Summarization on advantages and disadvantages of node indexing (Page 8),
	which I consider  is very important and useful in this paper.
	
	
	\subsection{Conclusion}
	
	This paper does not propose any new idea of indexing but introduces and compares
	five indexing techniques. Although it is a good paper for learning the indexing
	techniques and would also be useful for my own story, it is possibly not the one
	that I will choose to compare with.
	
	
	\section{Multi-Core Processing of XML Twig Patterns}
	
	\subsection{Background}
	
	XML is based on a tree-structured data model. Naturally, the most popular XML
	querying language (XPath) uses patterns of selection predicates, on multiple
	elements related by a tree structure, which often may be abstracted by twig
	patterns. Finding all occurrences of such a twig pattern in an XML database is a
	basic  operation for XML query processing. 
	
	This paper is about parallel twig matching algorithms. Two existing twig
	algorithms: Path Stack (PS) and Twig Stack (TS) are extended to parallel version,
	called PPS and the PTS proposed for matching XML query twig patterns in a
	parallel multi-threaded computing platform.
	
	\subsection{Technical Problems}
	
	How to improve the performance of processing twig patterns, specifically Twig
	Stack and Path Stack algorithms in terms of parallelization.
	
	\subsection{Technical Claims}
	
	This paper presents the parallel version of Path Stack algorithm (PPS) and Twig
	Stack algorithm (PTS). PPS and PTS are novel and efficient algorithms  for
	matching XML query twig patterns in a parallel multi-threaded  computing
	platform that are based on the PathStack and TwigStack algorithms. 
	
	
	\subsection{Discussion}
	
	This paper give definitions of twig pattern and path pattern. A \emph{twig pattern} 
	is defined as a rooted, ordered, labeled tree whose nodes’ labels are either
	element tags or string values, and edges are either parent-child edges or
	ancestor-descendant edges. A \emph{path pattern} is a special case of Twig pattern in
	which each node has at most one child. (I consider twig patern as a subset of XPath).
	
	These algorithms employ a sophisticated search technique for limiting processing
	to specific subtrees. The parallelization of the two existing algorithms employs
	a sophisticated search technique for limiting processing to specific subtrees so 
	that  query processing can be optimized.
	
	Experimental results indicate that using PPS and PTS significantly reduces the
	running time of queries in comparison with the PathStack/TwigStack algorithm (up
	to 44 times faster for DBLP queries and up to 22 times faster for XMark
	queries).
	
	\subsection{Conclusion}
	
	This paper proposes two novel algorithms PPS and PTS based on two existing
	algorithms that are used for efficiently processing twig matching. The 
	background and the goal are related to our study. Although twig pattern is a 
	bit different from XPath, it is still acceptable to consider as a subset of 
	XPath, making the study relevant to our study. We can refer this study for 
	our story.
	
	 
	
	
\end{document}